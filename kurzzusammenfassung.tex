% Kurzzusammenfassung nach den Vorgaben des Wilhelm-Ostwald-Gymnasiums
\documentclass[ngerman,geometry,paper=a4,fontsize=12pt]{scrreprt}

\usepackage[utf8]{inputenc}
\usepackage[T1]{fontenc}
%\usepackage{lipsum} % Für Beispieltext, kann entfernt werden zusammen mit dem lipsum-Kommando.
\usepackage{setspace}
\usepackage{csquotes}
\usepackage[ngerman]{babel}
\begin{document}
\thispagestyle{empty}
\onehalfspacing
\section*{Transkription von Vorlesungsaufnahmen für Question Answering mittels großer generativer Sprachmodelle}

Während der Corona-Pandemie wurden die meisten Vorlesungen der Universität Leipzig online gehalten. 
Bild- und Tonaufnahmen dieser Vorlesungen stehen auch heute noch zur Verfügung, sind allerdings schlecht zugänglich. 
Zum einen kann man sie nicht gezielt nach Inhalten durchsuchen und zum anderen ist die Veröffentlichung der Aufnahmen datenschutzrechtlich schwierig, da personenbezogene Daten über Studierende enthalten sind.\\ 
Ein Ansatz dieses Wissen verfügbar zu machen, bei dem diese Kriterien erfüllt werden, ist die automatisierte Fragebeantwortung mittels Künstlicher Intelligenz.
Dafür werden Aufnahmen des Sommersemesters 2021 aus dem Modul \enquote{Architektur von Informationssystemen im Gesundheitswesen} am Institut für Medizinische Informatik, Statistik und Epidemiologie (IMISE) verwendet. 
Als Basis wird ein anonymisiertes Transkript der fachlichen Inhalte verwendet. 
Dieses wird mit einem Transkriptionsmodell aus den Tonaufnahmen geschrieben und danach anonymisiert und von unwichtigen Inhalten gesäubert.
Ein weiteres Sprachmodell nutzt diese Daten um spezifisches Wissen wiederzugeben.
Studierende der Medizinischen Informatik können dadurch ihre Fragen schnell und eigenständig beantworten.
Dieser Vorgang soll auch für Vorlesungsaufnahmen anderer Forschungsbereiche wiederholt werden.\\\\
Simon Boldt


%\lipsum[1-2] % Beispieltext, löschen und durch eigenen Text ersetzen
\end{document}
