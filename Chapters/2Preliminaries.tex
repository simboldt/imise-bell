%*****************************************
\chapter{Grundlagen}\label{ch:preliminaries}
%*****************************************

\section{Medizinische Informatik}
\subsection{Informationssysteme im Gesundheitswesen}

\section{Transkription}



%Das Grundlagenkapitel soll den Stand der Forschung erläutern und mit Literatur belegen.
%Auf diesem Kapitel bauen die Erkenntnisse der Arbeit auf.
%Gerade in den Grundlagen wird man häufig Quellen benennen, aus denen die Aussagen letztlich stammen.
%Im Kapitel \enquote{Literaturverzeichnis} dieser Vorlage wird beschrieben, wie eine Quellenangabe zu erfolgen hat.

%Da sich die Medizinische Informatik mit der Lösung medizinischer Probleme befasst, sollen hier auch die Hintergründe des medizinischen Problems so dargestellt und erläutert werden, dass sie auch für Leser der Arbeit, die nicht Mediziner sind, verständlich sind.

%In diesem Kapitel werden auch die Methoden erläutert, die zur Lösung der Probleme eingesetzt wurden.
%Stellen Sie sicher, dass hier alle, aber auch nur die Grundlagen und Methoden erläutert werden, die in der Arbeit verwendet wurden.
%Stellen Sie im weiteren Text der Arbeit auch sicher, dass der Leser erkennen kann, wie Sie unter Verwendung der Methoden zu Ihren Ergebnissen gekommen sind.
%So sollten z.B. Modellierungsmethoden nur verwendet werden, wenn die Modelle nachvollziehbar dazu genutzt werden, die Ergebnisse zu erzielen.

%Bedenken Sie, dass Sie diese Arbeit zum Abschluss eines umfangreichen Studiums schreiben, das vor allem dazu diente, sie mit einem reichen Methodenrepertoire auszustatten.
%Wählen Sie aus den Methoden, die Sie gelernt haben, aus, benennen Sie die Methoden korrekt und wenden Sie sie an! Aber gehen sie auch kritisch mit dem um, was Ihnen gelehrt wurde.
%Wenn Sie feststellen, dass gelehrte Methoden ungeeignet sind, diskutieren Sie dies und suchen passendere Methoden! Wenn Sie Methoden benötigen, die nicht gelehrt wurden, suchen Sie nach passenden Methoden oder -- wenn Sie nicht fündig werden -- entwickeln Sie die für Ihr Problem passende Methode selbst!

%\begin{definition}[Beispieldefinition]
%Dies ist eine Beispieldefinition.
%\end{definition}

\section{Question Answering}

\section{Automatische Spracherkennung (ASR)}

\section{Neuronale Netze}
Neuronale Netze basieren auf der Funktionsweise biologischer Nervenzellen und werden verwendet um verschiedene komplexe Aufgaben zu lösen.
Sie bestehen aus vielen künstlichen Neuronen in verschiedenen Schichten.
Es gibt eine Eingabeschicht eine Ausgabeschicht und dazwischen beliebig viele versteckte Schichten.
Zwischen den Neuronen gibt es gewichtete Verbindungen (Abb 1 noch einfügen)
Den Neuronen der Eingabeschicht wird durch die Eingabe ein Wert zugewiesen.
Die Werte der weiteren Neuronen werden aus denen der vorherigen Schicht berechnet.
Dafür werden diese jeweils mit dem Gewicht der Verbindung zwischen den Neuronen Multipliziert und dann addiert.
Es kann auch ein Bias (beliebiger fester Wert) addiert werden.
Das Ergebnis dieser Rechnung wird durch eine weitere Funktion bearbeitet.
Die werte der Neuronen in der Ausgabeschicht bilden die Ausgabe des neuronalen Netzes.



\subsection{Trainig von neuronalen Netzen}#
Die Gewichte der Verbindungen und der Bias bilden die Parameter des neuronalen Netzes.


\subsection{Quantisierung}

\section{Sprachmodelle}
\subsection{Transformers}
\subsection{Bewertung von Transkriptionsmodellen}
Der gängigste Methode zur Bewertung der Qualität von Transkriptionsmodellen ist die \ac{wer}~\citep{wer}.
Diese gibt an, wie viele Prozent der Wörter N verglichen mit einem korrekten Transkript fehlerhaft sind.
Fehler sind dabei zusätzliche Wörter I (insertion), fehlende Wörter D (deletion) und ausgetauschte Wörter S (substitution):
\[\textnormal{WER} = \frac{I + D + S}{N} \times 100\]
Die \ac{wer} ist somit ein Maß für die Ungenauigkeit einer Transkription, wobei eine niedrigere \ac{wer} eine hohe Genauigkeit angibt.
Um sie möglichst genau zu ermitteln, müssen lange Audioeingaben verwendet werden.
Die Performance eines Transkriptionsmodells ist stark abhängig von der verwendeten Audio und den verwendeten Gräten, weshalb sich \acp{wer} aus verschiedenen Tests oft nicht vergleichen lassen.

\section{Pseudonymisierung und Anonymisierung}
Pseudonymisierung und Anonymisierung werden genutzt um wenn personenbezogene Informationen niemandem Zugeordnet werden sollen.
Es sind Wege solche Daten von Personen zu trennen um diese nicht zu belasten.

\begin{definition}[Pseudonymisierung]
D E F I N I T I O N
%Bei Pseudonymisierung werden Name von Personen durch andere Namen ersetzt.
%Diese Namen werden dann ganz normal Verwendet.
%Pseudonymisierung ist sinnvoll, wenn man zwischen Personen unterscheiden können muss ohne diese mit ihrem echten Namen zu nennen.
\end{definition}

\begin{definition}[Anonymisierung]
D E F I N I T I O N
%Bei Anonymisierung werden Namen von Personen entfernt. Bspw. wird dann "Max Mustermann" zu "Schüler"
\end{definition}