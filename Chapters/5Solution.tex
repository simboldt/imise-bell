%*****************************************
\chapter{Ausführung der Lösung}\label{ch:solution}
%*****************************************

\section{Transkription}

\subsection{Inhalt der Vorlesungen}

Eine Vorlesung besteht in der Regel aus einer Doppelstunde von 90 Minuten.
Hier wurden meistens gleich zwei Vorlesungen in Einheiten von etwa drei Stunden gehalten.
So ergeben sich 12 Aufzeichnungen mit insgesamt 22 Vorlesungen und einer Gesamtlänge von 32:05 Stunden. 
Der Inhalt dieser Vorlesungen richtet sich nach \citet{bb2}, siehe \cref{tab:vorlesungen}.

\begin{table}
\begin{tabulary}{\textwidth}{lLll}
\toprule
\textbf{Nr.} & \textbf{Kapitel} & \textbf{Datum} & \textbf{Länge}\\
\midrule
1 & 1. Introduction &  & \\
2 & 2. Health Institutions and Information Processing &  & \\
1--2 &  & 13.04.2021& 3:03:54\\
\midrule
3 & 3. Information System Basics & 15.04.2021 & 1:36:19 \\
\midrule
4 & 4. Health Information Systems &  & \\
 & 5. Modeling HIS &  & \\
5 & 5.1--5.2 &  & \\
4--5 &  & 20.04.2021 & 2:40:33 \\
\midrule
6--7 & 5.3--5.5 & 27.04.2021 & 2:56:34 \\ %26:52
\midrule
 & 6. Architecture of HIS &  & \\
8--9 & 6.1.--6.3. Functions, Data & 04.05.2021 & 3:00:36 \\
\midrule
10--11 & 6.4. Application Components& 11.05.2021 & 2:39:46 \\
\midrule
12--13 & 6.5. Integration & 18.05.2021 & 3:26:06 \\
\midrule
14--15 & 6.5. Integration & 25.05.2021 & 1:48:54 \\ %Ende Fehlt, Für eine kurze Pause ausgeschaltetr, vergessen
\midrule
16--17 & 6.6--6.7 Physical Tool Layer& 01.06.2021 & 3:11:34 \\
\midrule
18--19 & 7. Specific Aspects for Architectures of Transinstitutional HIS & 08.06.2021 & 3:17:44 \\
\midrule
20--21 & 8. Quality of HIS & 22.06.2021 & 2:50:15 \\
\midrule
22 & Ergänzung zu rechtlichen Anforderungen & 08.07.2021 & 1:32:45 \\
\midrule
& & & Gesamt: 32:05:00 \\
\bottomrule
\end{tabulary}
\caption{Zuordnung der Kapitel des Buchs zu den Vorlesungen}
\label{tab:vorlesungen}
\end{table}

\subsection{Transkriptionsmodell}

Im Buch 42k Wörter
Im Transkript 167k Wörter


Modell: Open AI Whisper \citep{whisper}
(Whisper: Open Source, Große Trainingsdatenmenge 680k Stunden, auf gute zero shot performance abgezielt -> kein weiteres Trainig erforderlich)

\subsubsection{Modell auf Website}

- playground.deepgram.com

Parameter
- Smart Format: Verbesserte Formatierung mit Erkennung von Zeit- oder Datumsangaben und Zahlen
- Punctuation: Satzzeichen
- Paragraphs: Transkript wird in Absätze unterteilt
- Diarization: Es wird zwischen verschiedenen Sprechenden unterschieden
Für Diarization kann Audio nicht gesplittet werden -> Einzelne Audiodatein mussten zu einer verbunden werden (ffmpeg)

Bei einzelnen Vorlesungen: Funktioniert grundsätzlich, Diarization inkonsistent, Genauer Vorgang unklar (Blackbox)

Programm hat für 10 min geladen, dann Abgestürzt aufgrund eines Netzwerkfehlers.
-> Lokale Whisper Anwendung bevorzugt

%https://platform.openai.com/docs/guides/speech-to-text
%https://openai.com/index/whisper/ 
%https://cdn.openai.com/papers/whisper.pdf

\subsubsection{API in eigenem Programm}

Whisper API in Python 
Problem: Immer nur 25MB gleichzeitig, output.opus  = 655 MB; keine Diarization 
Lösung: Datei in Abschnitte gliedern (OpenAIs Guide folgend)
Vorteil: Mehr Kontrolle durch Eingabe von Kontext

Sehr geringes limit mit der API (bei 30s testaudio schon überschritten)
Fehler: openai.RateLimitError: Error code: 429

\subsubsection{Modell vollständig lokal}
Canary  ->

