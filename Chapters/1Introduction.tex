
%************************************************
\chapter{Einleitung}\label{ch:introduction}
%************************************************

\section{Gegenstand}

Medizininformatik ist ein Masterstudiengang an der Fakultät der Mathematik und Informatik und der Medizinischen Fakultät der Universität Leipzig.
Während der Corona-Pandemie\footnote{Sommersemester 2020 bis einschließlich zum Wintersemester 2021/2022.} wurden die Vorlesungen dieses Studiengangs (damals Schwerpunktes) von Prof. Dr. Alfred Winter online über Big Blue Button gehalten.
Von diesen Vorlesungen liegen Aufnahmen der Präsentationen, Webcam-Mitschnitte, Audioaufnahmen und Chats vor. Weil die Universität Leipzig von Big Blue Button zu DFN-Conf gewechselt ist, mussten diese Aufnahmen gesichert werden.

\section{Problemstellung}

Aktuell haben Studierende der medizinischen Informatik wenige Möglichkeiten zur Beantwortung von spezifischen Fragen.
Bücher und Mitschriften verfügen nicht über alle Inhalte und müssen lange nach Antworten durchsucht werden.
Oft müssen dann andere Studierende oder Professoren zur Hilfe herangezogen werden, was auch nicht immer möglich ist.
Das Finden von Antworten auf fachspezifische Fragen zum Studiengang Medizininformatik ist also umständlich und zeitaufwendig.

%Problem P2: Bei der Veröffentlichung von Material muss das Institut Datenschutzrichtlinien einhalten.

\section{Motivation}

In den Vorlesungsaufnahmen ist viel nützliches Wissen über medizinische Informatik gespeichert.
Studierende brauchen eine Möglichkeit auf dieses Wissen zuzugreifen
Wenn ein gutes Transkript vorliegt, können sie Question Answering durch ein \acp{llm} nutzen, um so zu jeder Zeit schnell Antworten auf konkrete Fragen zu finden, was anderenfalls sehr aufwändig wäre. Dadurch könnte Lernfortschritt beschleunigt werden.


\section{Zielsetzung}\label{sec:zielsetzung}

Um das Problem zu lösen werden folgende Ziele angestrebt:


\begin{itemize}
\item Ziel Z1: Fehlerfreies Transkript nur der fachlichen Inhalte unter Einhaltung der Datenschutzbestimmungen~\citep{parlament2016verordnung}
\item Ziel Z2: Beantwortung von Fragen mittels \acp{llm} beispielhaft
\end{itemize}


Anspruch dieser Arbeit ist es Z1 zu erfüllen und somit Z2 vorzubereiten.
Sie soll eine Basis für die Bearbeitung von Z2 bilden.
%Die erklährung muss noch ausgefeilt werden                                                                                                                                                                 

\section{Aufgabenstellung}

Um Ziel Z1 zu erreichen müssen die Aufnahmen transkribiert werden.
Dabei sollen folgende Aufgaben durchgeführt werden.

\begin{itemize}
\item Aufgaben zu Ziel Z1:
	\begin{itemize}
	\item Aufgabe A1.1: Beschaffung und Anpassung der Aufnahmen
	\item Aufgabe A1.2: Transkriptionsmodell und Parameter auswählen
	\item Aufgabe A1.3: Transkript erstellen
	\item Aufgabe A1.4: Transkript korrigieren
	\item Aufgabe A1.5: Transkipt anonymisieren/pseudonymisieren
    \item Aufgabe A1.6: Entfernen von Passwörtern für Zugangsbeschränkte Lehrmaterialien
	\end{itemize}

%\item Aufgaben zu Ziel Z2:
%	\begin{itemize}
%	\item Aufgabe A2.1: Auwahl einer Methode zur Beantwortung von Fragen mithilfe eines Sprachmodells 
%	\item Aufgabe A2.2: Auswahl eines Sprachmodells
%	\item Aufgabe A2.3: Exemplarische Ausführung der Methode auf das Transkript
%	\end{itemize}
\end{itemize}
