%*****************************************
\chapter{Diskussion}\label{ch:discussion}
%*****************************************

%\section{Alternative Transkriptionssoftware}

%\section{Transkriptionsmethode}

\section{Verwendung von RAG als Question Answering Methode}
Auch wenn kein \ac{qa} in dieser Arbeit durchgeführt wird, spielt es dennoch eine Rolle für die Gestaltung des Transkripts.
Der Erstellungsprozess ist nämlich darauf ausgelegt, ein Transkript zu erschaffen, welches gut für \ac{qa} mittels \ac{rag} genutzt werden kann.
Die Verwendung von \ac{rag} ist dabei empfehlenswert, da so die Präzision der Antworten eines \ac{qa}-Modells erhöht wird.
Zusätzlich sind die Antworten verifizierbar, da erkennbar ist, welche Quelle verwendet wurde.
Die Speicherung des Transkripts im Kontext (Gedächtnis) eines Sprachmodells bietet, neben \ac{rag}, eine weitere Möglichkeit zur Einbeziehung zusätzlichen Wissens.
Jedoch sind die Kontextfenster von Sprachmodellen limitiert, wodurch sie nur eine begrenzte Menge an Daten gleichzeitig fassen können.
Im Gegensatz dazu, wird mit \ac{rag} nur ein kleiner Teil aus einer quasi beliebig großen Datenmenge ausgewählt und in die Beantwortung von Fragen einbezogen.
Außerdem erfordert \ac{rag} weniger Rechenleistung.
Eine andere Möglichkeit, um die Präzision eines \acp{llm} für einen bestimmten Themenbereich zu erhöhen, ist das Fine-Tuning.
Dabei wird ein \ac{llm} nachträglich mit Daten aus dem jeweiligen Bereich trainiert.
Das Trainieren eines Sprachmodells ist allerdings wesentlich komplizierter als die Anwendung eines bereits trainierten Modells.
Zudem werden große Mengen an Trainingsdaten benötigt, welche nicht für jeden Themenbereich zur Verfügung stehen.
Diese Methode wäre somit wesentlich aufwendiger und wird dem Anspruch einer einfachen übertragbaren Lösung nicht gerecht.


\section{Einsatz von KI im Bildungssystem}
Die in dieser Arbeit beschriebene Methode reiht sich ein in einen aktuellen wahrnehmbaren Trend von KI-basierten Anwendungen, die sich an Schüler und Studierende richten.
Viele davon sollen das Lösen unterschiedlicher Aufgaben erleichtern oder sogar übernehmen, was mitunter den Lerneffekt verringert und die Abhängigkeit von Technologie verstärkt.
Hausaufgaben können dadurch ohne viel Aufwand, vor allem aber ohne eigene Involvierung gelöst werden.
Dadurch geht deren Sinn Schüler zu fordern und einen Lern- oder Übungseffekt zu erbringen verloren. 
Das stellt eine Herausforderung für die Gestaltung solcher Aufgaben dar und das Bildungswesen muss sich dementsprechend anpassen.
Es gibt jedoch auch Anwendungsmöglichkeiten, welche beim Lernen und Üben unterstützen, indem sie z.B. Übungsaufgaben erstellen und korrigieren oder komplexe Themen verständlich zusammenfassen und erklären.
Auch durch KI-gestützte Übersetzung fremdsprachiger Texte oder durch Transkription von Audio-Quellen können verschiedene Inhalte besser zugänglich gemacht und zum Lernen genutzt werden.
Auch Lehrer können KI sinnvoll einsetzen, beispielsweise zur Gestaltung von Unterrichtsmaterial.


KI bringt viele neue Möglichkeiten Bildung effizienter zu gestalten.
Es ist wichtig, dass diese Chancen gut genutzt werden. 
