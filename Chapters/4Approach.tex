%*****************************************
\chapter{Lösungsansatz}\label{ch:approach}
%*****************************************

Diese Arbeit verfolgt das Ziel das Question Answering mittels \ac{rag} auf Basis eines Transkripts von Vorlesungsaufnahmen vorzubereiten.
Als erstes müssen dafür die Vorlesungen transkribiert werden.

\section{Transkription}

\subsection{Inhalt der Vorlesungen}
Eine Vorlesung besteht in der Regel aus einer Doppelstunde von 90 Minuten.
Meistens wurden gleich zwei Vorlesungen in Einheiten von etwa drei Stunden gehalten.
So ergeben sich 12 Aufzeichnungen mit insgesamt 22 Vorlesungen und einer Gesamtlänge von 32 Stunden und fünf Minuten. 
Der Inhalt dieser Vorlesungen basiert auf \cite{bb2}. % Dieses Buch behandelt ... %Hier noch HIS, Inhalt des Buchs beschreiben
Die Zuordnung der Kapitel des Buchs zu den Vorlesungen ist in \cref{tab:vorlesungen} dargestellt.

\begin{table}
\begin{tabulary}{\textwidth}{lLll}
\toprule
\textbf{Nr.} & \textbf{Kapitel} & \textbf{Datum} & \textbf{Länge}\\
\midrule
1 & 1. Introduction &  & \\
2 & 2. Health Institutions and Information Processing &  & \\
1--2 &  & 13.04.2021& 3:03:54\\
\midrule
3 & 3. Information System Basics & 15.04.2021 & 1:36:19 \\
\midrule
4 & 4. Health Information Systems &  & \\
 & 5. Modeling HIS &  & \\
5 & 5.1--5.2 &  & \\
4--5 &  & 20.04.2021 & 2:40:33 \\
\midrule
6--7 & 5.3--5.5 & 27.04.2021 & 2:56:34 \\ %26:52
\midrule
 & 6. Architecture of HIS &  & \\
8--9 & 6.1.--6.3. Functions, Data & 04.05.2021 & 3:00:36 \\
\midrule
10--11 & 6.4. Application Components& 11.05.2021 & 2:39:46 \\
\midrule
12--13 & 6.5. Integration & 18.05.2021 & 3:26:06 \\
\midrule
14--15 & 6.5. Integration & 25.05.2021 & 1:48:54 \\ %Ende Fehlt, Für eine kurze Pause ausgeschaltetr, vergessen
\midrule
16--17 & 6.6--6.7 Physical Tool Layer& 01.06.2021 & 3:11:34 \\
\midrule
18--19 & 7. Specific Aspects for Architectures of Transinstitutional HIS & 08.06.2021 & 3:17:44 \\
\midrule
20--21 & 8. Quality of HIS & 22.06.2021 & 2:50:15 \\
\midrule
22 & Ergänzung zu rechtlichen Anforderungen & 08.07.2021 & 1:32:45 \\
\midrule
& & & Gesamt: 32:05:00 \\
\bottomrule
\end{tabulary}
\caption{Zuordnung der Kapitel des Buches zu den Vorlesungen}
\label{tab:vorlesungen}
\end{table}

\subsection{Methode}
Die Transkription soll ein \ac{llm} verwendet werden.
%Spart enorm Zeit, ist sogar genauer als transkription per Hand (Paper als Quelle)
Aufgrund seiner hohen Genauigkeit, wurde OpenAI Whisper als Transkriptionsmodell ausgewählt.
Insbesondere beim Umgang mit Fachsprache übertrifft Whisper andere Modelle.
Da in den Vorlesungen vor allem fachliche Inhalte thematisiert werden ist diese Eigenschaft besonders wichtig.
Aus den verschiedenen Whisper Modellen wurde Whisper turbo ausgewählt, weil es eines der genausten und gleichzeitig schnellsten ist.
%Jetz noch noScribe (ausgewählt weil open source, whisper turbo voreingestellt)
%RAG (cite\hallucination)
%warum exemplarisch ok ist

%~~~~~~~~~~Scheitern von OpenAI API~~~~~~~~~~~~~~~
%https://platform.openai.com/docs/guides/speech-to-text

%Whisper API in Python 
%Problem: Immer nur 25MB gleichzeitig, output.opus  = 655 MB; keine Diarization 
%Lösung: Datei in Abschnitte gliedern (OpenAIs Guide folgend)

%Sehr geringes limit mit der API
%Fehler: openai.RateLimitError: Error code: 429
