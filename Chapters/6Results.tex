%*****************************************
\chapter{Ergebnisse}\label{ch:results}
%*****************************************

Die Transkription mit Whisper wurde für die ausgewählten Vorlesungen aus dem Modul \enquote{Architektur von Informationssystemen im Gesundheitswesen} erfolgreich durchgeführt
Dabei insgesamt 32 Stunden und fünf Minuten an Tonmaterial innerhalb von etwa sechs Stunden in Textform transkribiert.
Dieser Vorgang dauerte somit nur etwa ein Fünftel der Gesamtlänge der Vorlesungen.
Als Ergebnis liegt ein Transkript von knapp 170.000 Wörtern Umfang vor.
Die manuelle Überarbeitung dieses Transkripts erfolgte beispielhaft für die Vorlesung \enquote{Information System Basics} (Nr.3).
Dieser Teil befindet sich im Anhang und ist jetzt für \ac{qa} mittels \ac{rag} nutzbar.

\section{Einschränkungen}

Die erste Einschränkung liegt in der Genauigkeit des Transkripts.
Kompliziertes Vokabular wie zum Beispiel spezifische Fachbegriffe, Eigennahmen oder Abkürzungen werden nicht immer korrekt erkannt und dadurch Falsch wiedergegeben.
Das liegt daran, dass solche Ausdrücke nicht in den Trainingsdaten der meisten \acp{llm} enthalten sind und Schreib- und Sprechweise teilweise nicht übereinstimmen.
Zudem können, ähnlich wie bei einem menschlichen Transkribierer, auch bei der Transkription mit einem \ac{llm} undeutliche Aussprachen zu Fehler führen.
Durch eine Korrektur durch einen Menschen mit Fachkenntnissen lassen sich Transkriptionsfehler jedoch größtenteils beheben.
Weitere Einschränkungen bestehen jedoch auch bei einer vollständig korrekten Transkription.
Wenn sich ein Sprecher auf Inhalte in einer Präsentation bezieht, kann er annehmen, dass die Teilnehmer der Vorlesung die Bildquelle sehen.
Dadurch, dass nur das Audiomaterial der Vorlesungen berücksichtigt wird geht dann ein Teil der Informationen verloren.
Außerdem ist die Sprechweise in einer Vorlesung oftmals etwas umgangssprachlich und enthält Füllwörter und Abschweifungen.
Dadurch sind die Inhalte für einen menschlichen Zuhörer anschaulicher, allerdings wird das Gesagte in Schriftform schnell unübersichtlich.
Zudem passieren mündlich gelegentlich Versprecher, welche den Inhalt verfälschen können.
Es ist zu erwarten, dass dadurch \ac{qa} auf Basis des Transkripts leicht beeinträchtigt wird.


\section{Fazit}

Aufnahmen von Studienvorlesungen enthalten ein bisher schwer zugängliches und daher wenig genutztes Wissenspotenzial.
Mit dieser Arbeit wird eine Methode vorgestellt, mit der dieses Wissen effizient durch Nutzung von \acp{llm} erschlossen werden kann.
Die Methode ist ohne spezifische Anpassungen auf jegliche Audio- und auch auf Videoaufnahmen übertragbar,
Dadurch kann sie analog bei Aufnahmen anderer Themenbereiche angewendet werden.

%(Vergleich mit Programmbeschreibung, 2-5 mal langsamer, beschleunigt durch CUDA) 
%- Genauigkeit: noch zu ermitteln