%*****************************************
\chapter{Stand der Forschung}\label{ch:relatedWork}
%*****************************************

%Kann auch im Grundlagenkapitel mit integriert werden! Wenn es zwei Kapitel sind, ist das Grundlagenkapitel für bekanntes \enquote{Lehrbuchwissen} gedacht, der \enquote{Stand der Forschung} für Vorarbeiten in Publikationen.

%Grundlage für eine Abschlussarbeit ist eine gründliche Recherche im Themenumfeld.
%Dabei ist es ausdrücklich nicht hinreichend, mit bekannten Suchmaschinen im Internet zu recherchieren.
%Vielmehr wird von den Studierenden erwartet, dass sie auch referierte Veröffentlichungen (wissenschaftliche Zeitschriften (auch elektronisch), Bücher) in die Erarbeitung einbeziehen (und mit entsprechenden Quellenangaben belegen).
%Informationen zum Thema der Literaturrecherche finden sich in den \href{http://www.imise.uni-leipzig.de/Lehre/MedInf/Abschlussarbeiten/Literaturrecherche}{Hinweisen zur Literaturrecherche}.
%Wir empfehlen Ihnen, Kurse zur Literaturrecherche zu belegen, z.B. beim \href{https://home.uni-leipzig.de/academiclab/}{Academic Lab} oder der \href{https://www.ub.uni-leipzig.de/service/workshops-und-online-tutorials/}{Universitätsbibliothek}.
%Im \href{https://home.uni-leipzig.de/schreibportal/}{Onlineportal zum Wissenschaftlichen Schreiben der Uni Leipzig} finden Sie wertvolle Hinweise und Übungen zu Textstruktur, Stilistik, Schreibprozess sowie Quellen und Zitaten.

%\paragraph{Beispielzitierungen}
%\citet{sniktec} beschreiben ein Verfahren zur X von Y auf Basis von Z.
%Alternativ: X von Y lässt sich auf Basis von Z ermitteln~\citep{sniktec}.

\section{Transkriptionsmodelle}
\subsection{Bewertung von Transkriptionsmodellen}
Der gängigste Methode zur Bewertung der Qualität von Transkriptionsmodellen ist die \ac{wer}~\citep{wer}.
Diese gibt an, wie viele Prozent der Worte N verglichen mit einem korrekten Transkript fehlerhaft sind.
Fehler sind dabei zusätzliche Wörter I (insertion), fehlende Wörter D (deletion) und ausgetauschte Wörter S (substitution):
\[\textnormal{WER} = \frac{I + D + S}{N} \times 100\]
Die \ac{wer} ist somit ein Maß für die Ungenauigkeit einer Transkription, wobei eine niedrigere \ac{wer} eine hohe Genauigkeit angibt.
Um sie zu möglichst genau zu ermitteln, müssen lange Audiodateien verwendetet werden.
Die Performance eines Transkriptionsmodells ist stark Abhängig von der verwendeten Audio und den, weshalb sich \acp{wer} aus Verschiedenen Tests oft nicht vergleichen lassen.


\subsection{Whisper}
Es existieren verschiedene Whisper Modelle (\cref{tab:whisper_modelle})
Das größte dieser Modelle, Whisper large, gibt es in drei Versionen.
Whisper large, Whisper large-v2, Whisper large-v3.
Whisper turbo ist eine optimierte Version von Whisper large-v3.
Die Modelle werden mit zunehmender Größe genauer, aber auch langsamer.
Whisper Turbo ist dabei eine Ausnahme, weil es ähnlich genau wie Whisper large-v2 und gleichzeitg fast so schnell wie Whisper tiny ist.

\begin{table}
\begin{tabulary}{\textwidth}{llll}
\toprule
\textbf{Modell} & \textbf{Parameter}
Tiny & 39M\\
Base & 74M\\
Small & 244M\\
Medium & 769M\\
Large & 1550M\\
Turbo & 809M\\
\bottomrule
\end{tabulary}
\caption{Whisper Modelle}
\label{tab:whisper_modelle}
\end{table}



 %Noch kein Paper dazu, Nicht zitierfähige Quellen dazu: https://huggingface.co/openai/whisper-large-v3-turbo, https://github.com/openai/whisper?tab=readme-ov-file, https://github.com/openai/whisper/discussions/2363

\citet{VergleichASR2023}
Vergleich vieler ASR Modelle anhand Deutsch- und Englischsprachiger Youtubevideo Ausschnitte: OpenAI Whisper in allen Kategorien niedrigste \ac{wer}, vor allem in Deutscher Sprache mit 5.0 bei einem Durchschnitt von 15.3 (Platz 2: Speechmatics 8.0, Platz 3: Microsoft 10.1), allerdings am Langsamsten



\subsection{Nvidia Canary}
\citet{canary}

\begin{table}
\begin{tabulary}{\textwidth}{lLll}
\toprule
\textbf{Modell} & \textbf{Parameter} & \textbf{WER (De)} & \textbf{Training}\\
Whisper-large v3 & 1,55b & 9,5 & 680k hours\\
Canary & 1b & 6,4 & 86k hours\\
\bottomrule
\end{tabulary}
\caption{Vergleich von Canary 1b und Whisper}
\label{tab:Canary vs Whisper}
\end{table}

\subsection{Moonshine}


\section{Nvidia CUDA}
