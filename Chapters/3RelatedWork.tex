%*****************************************
\chapter{Stand der Forschung}\label{ch:relatedWork}
%*****************************************

%Kann auch im Grundlagenkapitel mit integriert werden! Wenn es zwei Kapitel sind, ist das Grundlagenkapitel für bekanntes \enquote{Lehrbuchwissen} gedacht, der \enquote{Stand der Forschung} für Vorarbeiten in Publikationen.

%Grundlage für eine Abschlussarbeit ist eine gründliche Recherche im Themenumfeld.
%Dabei ist es ausdrücklich nicht hinreichend, mit bekannten Suchmaschinen im Internet zu recherchieren.
%Vielmehr wird von den Studierenden erwartet, dass sie auch referierte Veröffentlichungen (wissenschaftliche Zeitschriften (auch elektronisch), Bücher) in die Erarbeitung einbeziehen (und mit entsprechenden Quellenangaben belegen).
%Informationen zum Thema der Literaturrecherche finden sich in den \href{http://www.imise.uni-leipzig.de/Lehre/MedInf/Abschlussarbeiten/Literaturrecherche}{Hinweisen zur Literaturrecherche}.
%Wir empfehlen Ihnen, Kurse zur Literaturrecherche zu belegen, z.B. beim \href{https://home.uni-leipzig.de/academiclab/}{Academic Lab} oder der \href{https://www.ub.uni-leipzig.de/service/workshops-und-online-tutorials/}{Universitätsbibliothek}.
%Im \href{https://home.uni-leipzig.de/schreibportal/}{Onlineportal zum Wissenschaftlichen Schreiben der Uni Leipzig} finden Sie wertvolle Hinweise und Übungen zu Textstruktur, Stilistik, Schreibprozess sowie Quellen und Zitaten.

%\paragraph{Beispielzitierungen}
%\citet{sniktec} beschreiben ein Verfahren zur X von Y auf Basis von Z."
%Alternativ: X von Y lässt sich auf Basis von Z ermitteln~\citep{sniktec}.

\section{Transkriptionsmodelle}

\subsection{Whisper}
Whisper ist ein Open-Source-Transkriptionsmodell von OpenAI, welches mit 680.000 Stunden an Audio in verschieden Sprachen aus dem Internet trainiert wurde.
Aufgrund der weitreichenden Diversität der Trainingsdaten erreicht Whisper auch bei Akzenten und Fachsprache  hohe Genauigkeit \citep{whisper}.
Durch den großen Umfang des Trainings erreicht Whisper eine gute Zero-Shot-Performance, kommt also gut mit Daten zurecht, auf die es nicht spezifisch trainiert wurde.

%Vergleich vieler ASR Modelle anhand deutsch- und englischsprachiger Youtubevideo Ausschnitte: OpenAI Whisper in allen Kategorien niedrigste \ac{wer}, vor allem in deutscher Sprache mit 5.0 bei einem Durchschnitt von 15.3 (Platz 2: Speechmatics 8.0, Platz 3: Microsoft 10.1), allerdings am langsamsten \citet{VergleichASR2023}

\subsubsection{Whisper-Modelle}
Es existieren verschiedene Whisper-Modelle (\cref{tab:whisper_modelle})
Das größte dieser Modelle, Whisper large, gibt es in drei Versionen.
Whisper large, Whisper large-v2, Whisper large-v3.
Whisper turbo ist eine 8bit quantisierte Version von Whisper large-v3.
Die Modelle werden mit zunehmender Größe genauer, aber auch langsamer.
Whisper Turbo ist dabei eine Ausnahme, weil es ähnlich genau wie Whisper large-v2 und gleichzeitig fast so schnell wie Whisper tiny ist. \cite{distilwhisper}

\begin{table}
\begin{tabulary}{\textwidth}{lS}
\toprule
\textbf{Modell} & \textbf{Parameter}\\
Tiny & \qty{39}{\mega\nounit}\\
Base & \qty{74}{\mega\nounit}\\
Small & \qty{244}{\mega\nounit}\\
Medium & \qty{769}{\mega\nounit}\\
Large & \qty{1550}{\mega\nounit}\\
Turbo & \qty{809}{\mega\nounit}\\
\bottomrule
\end{tabulary}
\caption{Whisper-Modelle}
\label{tab:whisper_modelle}
\end{table}

%\subsection{Nvidia Canary}
%\citet{canary}
%
%\begin{table}
%\begin{tabulary}{\textwidth}{lLll}
%\toprule
%\textbf{Modell} & \textbf{Parameter} & \textbf{WER (De)} & \textbf{Training}\\
%Whisper-large v3 & 1,55b & 9,5 & 680k hours\\
%Canary & 1b & 6,4 & 86k hours\\
%\bottomrule
%\end{tabulary}
%\caption{Vergleich von Canary 1b und Whisper}
%\label{tab:Canary vs Whisper}
%\end{table}
%
%\subsection{Moonshine}


\section{Nvidia CUDA}
