%*******************************************************
% Summary
%*******************************************************
\pdfbookmark[1]{Zusammenfassung}{summary}
\chapter*{Zusammenfassung}
\addcontentsline{toc}{chapter}{Zusammenfassung}

%In der Zusammenfassung wird vor allem zusammenfassend dargestellt, welche Ergebnisse zu den formulierten Zielen erreicht wurden.
%Bei der Formulierung sollten wichtige Schlüsselbegriffe verwendet werden.
%Wurden Fragen formuliert, können auch diese hier beantwortet werden.
%So soll es möglich sein, dass ein Leser von der Arbeit lediglich die Einleitung und die Zusammenfassung lesen und doch die Ergebnisse der Arbeit erfassen kann.

%Checkliste:
%\begin{enumerate}
%\item Kurze, klare Darstellung zu Gegenstand, Fragestellung und Zielsetzung der Arbeit
%\item Angewandte Methodik
%\item Vorstellung der wesentlichen Ergebnisse und daraus resultierende Schlussfolgerungen.
%\end{enumerate}
%Für wissenschaftliche Abschlussarbeiten sind etwa 200 bis 250 Wörter ausreichend.

Während der Corona-Pandemie wurden die Vorlesungen des Masterstudiengangs Medizininformatik online gehalten. 
Die Aufnahmen dieser Vorlesungen stehen auch heute noch zur Verfügung sind allerdings schlecht zugänglich. 
Zum einen kann man sie nicht gezielt nach Inhalten durchsuchen und zum anderen ist die Veröffentlichung der Aufnahmen datenschutzrechtlich schwierig, da personenbezogene Daten enthalten sind. 
Zur Lösung beider Probleme wird ein anonymisiertes Transkript der fachlichen Inhalte erstellt. 
Mittels Question Answering können Studenten dann schnell Antworten auf konkrete Fragen finden. 

\vfill